\documentclass{article}
\usepackage{hyperref}
\usepackage{graphicx}
\usepackage{authblk}
\usepackage{amsmath}
\usepackage{float}
\usepackage{subcaption}
\usepackage{fancyhdr}
\usepackage{enumitem}
\usepackage{subcaption}
\usepackage{pgfplots}
\newtheorem{definition}{Definition}
\newtheorem{theorem}{Theorem}
\pgfplotsset{compat=1.18} 
\renewcommand*{\Authsep}{}
\renewcommand*{\Authand}{}
\renewcommand*{\Authands}{}

\usepackage[a4paper,margin=2.5cm]{geometry}

\setlist[itemize]{noitemsep,nolistsep}
\setlist[enumerate]{noitemsep,nolistsep}

\pagestyle{fancy}

\fancyhf{}
\fancyfoot[R]{\thepage}
\pagestyle{fancy}
\renewcommand{\headrulewidth}{0.4pt}%
\fancyfoot[R]{\thepage}

\lhead{RA-MIRI}

\title{Assignment 2 - Balanced allocations  \\ \vskip 1em \Large Randomized Algorithms}

\author[*]{Nicolás Zhao}

\affil[*]{Universitat Politècnica de Catalunya (UPC)}
\affil[*]{\textit {nicolas.zhao@estudiantat.upc.edu}}

\date{\today}

\begin{document}

\maketitle

\begin{center}
  \textbf{Source code} at \url{https://github.com/nicolasZhao1908/RA-balancedalloc}
\end{center}

\section{Introduction}

\begin{figure}[htbp]
  \centering
  \begin{subfigure}{0.4\textwidth}
    \centering
    \includegraphics[width=0.8\textwidth]{example-image-a}
  \end{subfigure}
  \begin{subfigure}{0.4\textwidth}
    \centering
    \includegraphics[width=0.8\textwidth]{example-image-b}
  \end{subfigure}
  \caption{Example figure}
  \label{fig:example}
\end{figure}


\end{document}
