\documentclass[11pt]{article}
\usepackage{hyperref}
\usepackage{graphicx}
\usepackage{authblk}
\usepackage{amsmath}
\usepackage{float}
\usepackage{subcaption}
\usepackage{fancyhdr}
\usepackage{enumitem}
\usepackage{pgfplots}
\newtheorem{definition}{Definition}
\newtheorem{theorem}{Theorem}
\pgfplotsset{compat=1.18} 
\renewcommand*{\Authsep}{}
\renewcommand*{\Authand}{}
\renewcommand*{\Authands}{}
\graphicspath{{./assets/}}

\usepackage[a4paper,margin=2.5cm, headheight=14pt]{geometry}

\setlist[itemize]{noitemsep,nolistsep}
\setlist[enumerate]{noitemsep,nolistsep}

\pagestyle{fancy}

\fancyhf{}
\fancyfoot[R]{\thepage}
\lhead{RA-MIRI}

\title{Assignment 2 - Balanced Allocations \\ \vskip 1em \Large Randomized Algorithms}

\author[*]{Nicolás Zhao}

\affil[*]{Universitat Politècnica de Catalunya (UPC)}
\affil[*]{\textit{nicolas.zhao@estudiantat.upc.edu}}

\newcommand{\batchplot}[4]{
    \begin{figure}[!htbp]
        \centering
        \begin{subfigure}{0.45\textwidth}
            \centering
            \includegraphics[width=\textwidth]{avg-#4_n-#1_batch-#2.pdf}
            \caption{Average - #1 batches}
        \end{subfigure}
        \hfill
        \begin{subfigure}{0.45\textwidth}
            \centering
            \includegraphics[width=\textwidth]{stddev-#4_n-#1_batch-#2.pdf}
            \caption{Standard Deviation - #1 batches}
        \end{subfigure}
        \caption{#3 performance for $n=#4$ and $b=#1$}
        \label{fig:#1-batch-#2-#4}
    \end{figure}
}

\date{\today}

\begin{document}

\maketitle

\begin{center}
  \textbf{Source code} available at \url{https://github.com/nicolasZhao1908/RA-balancedalloc}
\end{center}

\section{Introduction}

This study explores randomized allocation strategies for distributing a set of
requests, represented as "balls," across a set of resources or "bins," aiming
to minimize the maximum load difference—referred to as the \textit{gap}—across
bins. The core objective is to achieve balanced load distribution as the number
of requests increases, ensuring that no single bin becomes overly burdened. The
maximum gap, $G_n$, is defined as:
\[
G_n = \operatorname*{max}_{1 \le i \le m} \left\{ X_i(n) - \frac{n}{m} \right\},
\]
where $X_i(n)$ represents the load of bin $B_i$, indicating the deviation from
the ideal average load $\frac{n}{m}$ after distributing $n$ balls across $m$
bins.

This investigation focuses on three primary allocation strategies, each
employing different approaches to load data utilization and choice mechanisms:

\begin{itemize}
  \item \textbf{d-choice ($d \ge 1$):} Each ball is assigned to the
    least-loaded of $d$ randomly selected bins. This method improves load
    balance as $d$ increases, though it may be affected by outdated information
    in batched allocations.
  
  \item \textbf{(1+$\beta$)-choice:} A hybrid method that combines one-choice
    and two-choice strategies. With probability $\beta$, a one-choice strategy
    (random bin selection) is applied, and with probability $1 - \beta$, a
    two-choice strategy is used, where the least-loaded of two bins is chosen.
    This strategy balances the need for load information with flexibility,
    making it suitable for distributed settings where real-time load monitoring
    may be limited.

  \item \textbf{Partial Information:} This approach uses minimal load
    information, such as threshold-based queries indicating whether a bin’s
    load exceeds a certain level without specifying exact loads. This strategy
    is designed for environments where continuous, detailed load data is
    unavailable or costly to maintain.

\end{itemize}

We also examine the impact of \textit{batch allocation}, where groups of balls
are allocated simultaneously without updating load information during each
batch. Non-batched allocation is treated as a special case with a batch size of
1. By analyzing scenarios with different batch sizes and load levels—ranging
from light-loaded ($n = m$) to heavy-loaded ($n = m^2$)—we assess how each
allocation strategy influences the maximum gap, $G_n$, in diverse settings.

This study provides insights relevant to load balancing in distributed
computing and data center applications, where effective allocation strategies
are crucial for maintaining performance and managing resource distribution
under varying demand conditions. By examining choice-based and data-limited
methods, our findings offer guidance on selecting optimal strategies based on
specific constraints in distributed environments.

\section{Experiments}

In this section, we evaluate the effectiveness of various load balancing
strategies through a series of experiments, analyzing each method's performance
under different configurations and constraints. These methods include the
$d$-choices approach, the $(1+\beta)$-choices hybrid strategy, and the Partial
Information approach, each designed to manage load distribution with varying
levels of data availability and batch allocation sizes. Our experiments aim to
assess the impact of choice-based and data-limited strategies on the overall
balance of load distribution, particularly in settings with high request volume
or communication limitations.

The $d$-choices method examines the effect of increasing choice options, where
each allocation decision can sample multiple bins to choose the least-loaded
option, effectively improving load balance as $d$ increases. However, we
observe that outdated load information can negatively affect performance in
high-batch allocations, highlighting a key tradeoff between choice quantity and
data freshness.

The $(1+\beta)$-choices method, which blends one-choice and two-choice
strategies, is evaluated for its ability to balance load effectively with
limited but timely load data, especially in networked environments where
real-time load monitoring may not be feasible. This hybrid approach aims to
minimize load imbalance while limiting the reliance on up-to-date data, making
it particularly suitable for distributed systems with high request-to-server
ratios.

Finally, the Partial Information strategy is assessed as a practical
alternative for scenarios where access to complete load data is restricted.
This method leverages basic load threshold queries, which provide approximate
information on bin loads and guide allocation decisions without requiring
continuous real-time data. Partial Information offers a balance between minimal
data usage and effective load management, especially in distributed
environments where inter-node communication is constrained.

Through the analysis of each approach under varying batch sizes and load
conditions, we aim to identify the strengths and limitations of each strategy.
The following subsections present detailed experimental results, providing
insights into the applicability of these methods across different distributed
system settings and guiding the choice of allocation strategy based on specific
environmental constraints.

\subsection{d-Choices}

Figure \ref{fig:1-batch-d-choices-10000} demonstrates how the average and
standard deviation of gaps change as the number of choices $d$ increases. The
1-choice method, which does not use bin load information, performs the worst in
terms of load balancing. However, as dd increases, the allocation performance
significantly improves, with lower average gaps, even for relatively low dd
values compared to the number of bins mm.

For larger batch sizes, such as $b=5m$ and $b=20m$ (shown in Figures
\ref{fig:500-batch-d-choices-10000} and \ref{fig:2000-batch-d-choices-10000}),
higher dd-values lead to increased $\overline{G_n}$. This is due to the use of
outdated load information, causing some bins to accumulate more load than
others. Spikes in $\overline{G_n}$ for $d>1$ in these figures reveal that
once batch allocations begin, bins that initially have similar loads can end up
disproportionately loaded by the batch’s end. This occurs because each ball is
allocated to what appears to be the least-loaded bin based on old data, leading
to a ``swinging'' effect: one batch creates an imbalance, and in the following
batch, bins mistakenly chosen as minimum in hte previous batch become less
likely to be selected, and so on.

\foreach \i in {1, 500, 2000}{ 
  \batchplot{\i}{d-choices}{$d$-choices}{10000} 
}

For cases where batch size is very large ($b \gg m$), as shown in Figure
\ref{fig:5000-batch-d-choices-10000}, higher $d$-choices strategies eventually
perform worse than the 1-choice strategy. This is because 1-choice
always selecting bins randomly and do not rely on information of the chosen
bins, thus it performs always the same independent of the presence of outdated
information of bins.

\batchplot{5000}{d-choices}{$d$-choices}{10000} 


\subsection{\texorpdfstring{(1+$\beta$)}-Choices}

In Figure \ref{fig:1-batch-dchoice-vs-beta-10000}, we compare
$(1+\beta)$-choice strategies with $d$-choices under sequential allocation
($b=1$). Even a low frequency of two-choice decisions (e.g., $\beta=0.75$)
yields much better performance than the pure one-choice case, especially as $n$
approaches $m^2$. This is significant, as full real-time knowledge of each bin's
load may not be feasible in a networked system with limitations such as traffic
contention, rate limits, or latency.

The effectiveness of this hybrid approach becomes even clearer in batched
allocations with batch sizes $b=5m$, $b=20m$, and $b=50m$ (Figures
\ref{fig:500-batch-dchoice-vs-beta-10000},
\ref{fig:2000-batch-dchoice-vs-beta-10000}, and
\ref{fig:5000-batch-dchoice-vs-beta-10000}). In environments with more requests
(balls) than available servers (bins), distributing requests to servers at
random but occasionally checking their current status (around $25\%$ of the time
with $\beta=0.75$) significantly improves load balance.

\foreach \i in {1, 500, 2000, 5000}{ 
  \batchplot{\i}{dchoice-vs-beta}{$d$-choice vs $\beta$-choice}{10000} 
}

Across the tested values for $\beta \in \{0.25,0.5,0.75\}$, each
outperforms traditional $d$-choice strategies in batched settings, but the case
with $\beta=0.75$ yields the best results in highly parallel allocations. This
approach not only minimizes the average gap $\overline{G_n}$ for most values
of $n$ but also almost eliminates the swinging behavior seen in the 2-choice
strategy, leading to more stable and balanced load distributions over time.

\subsection{Partial Information}

The Partial Information strategy enables allocation decisions to be made without
precise knowledge of each bin's load, offering a practical approach when load
data is constrained due to factors such as network latency, data access
restrictions, or high communication costs. In scenarios where real-time,
detailed load data is not available, this method provides an efficient
alternative by relying on threshold queries to approximate load conditions.

Figures \ref{fig:1-batch-all-10000}, \ref{fig:500-batch-all-10000}, and
\ref{fig:2000-batch-all-10000} illustrate the performance of Partial Information
under various batch sizes. Even with basic threshold checks (e.g., $k=1$ or
$k=2$ queries, where $k$ represents the number of threshold levels queried),
this method achieves a load balance that is competitive with more data-intensive
strategies like $d$-choice. Notably, Partial Information demonstrates higher
stability, especially in high-load and batched scenarios, where each batch
contains a large number of allocations without intermediate load updates.

This approach holds particular significance in distributed computing
environments where access to comprehensive load information across all nodes is
not feasible. In such systems, nodes typically lack detailed insights into each
other’s current load but may have access to global system metrics or basic
threshold checks that allow them to make reasonable allocation choices. The
Partial Information strategy models this real-world limitation, aligning well
with scenarios where data on node states is restricted or sporadically
available.

In practice, Partial Information not only reduces the communication overhead
required to gather detailed load data but also achieves near-optimal load
distribution by focusing on identifying heavily loaded bins and diverting
allocations away from them. This method is especially beneficial in environments
with high parallelism or limited inter-node communication, where more frequent
data querying could otherwise lead to significant latency or congestion.

The impact of Partial Information is further evident in batched allocation
experiments, where load data is updated only after each batch completes. In
these cases, threshold-based decisions prevent severe imbalances that can arise
from using outdated information, as seen in high-$d$-choice scenarios with
larger batch sizes. As Figures \ref{fig:500-batch-all-10000} and
\ref{fig:2000-batch-all-10000} show, Partial Information maintains lower and
more consistent gap values, making it particularly effective for applications in
distributed systems where rapid, approximate load balancing decisions are
essential.

\foreach \i in {1, 500, 2000, 5000}{
  \batchplot{\i}{all}{Partial information vs others}{10000}
}

Overall, the Partial Information strategy bridges the gap between
information-intensive allocation methods and random strategies by leveraging
limited load indicators to maintain efficient balance. It is particularly well
suited for distributed environments that demand low communication overhead and
high scalability.

\section{Conclusion}

This study provides insights into how various allocation methods affect the
efficiency of load distribution in randomized settings. The main findings are
summarized as follows:

\begin{enumerate}
  \item \textbf{Choice-based methods}: Increasing the number of bin choices
    during allocation (i.e., higher $d$-values) typically enhances load
    balance, as it allows allocations to gravitate towards less-loaded bins.
    However, in batched allocations where load data may become outdated, an
    excessive $d$ can lead to imbalances, as earlier choices are made based on
    stale information.

  \item \textbf{Hybrid strategies}: The ($1+\beta$)-choice approach offers
    robustness in high-demand or networked environments where real-time load
    information is limited. By incorporating a probabilistic mix of one-choice
    and two-choice strategies, this method strikes a balance between performance
    and information requirements, minimizing the risk of severe load spikes
    in highly concurrent environments.

  \item \textbf{Partial Information utility}: In settings with restricted or
    delayed load data, the Partial Information strategy proves highly effective.
    By employing load threshold queries, this approach achieves competitive
    balance with minimal communication overhead, making it particularly valuable
    in distributed environments where nodes can only access limited information.
    This method aligns well with real-world distributed systems, where nodes
    generally lack comprehensive insights into other nodes' loads but may rely
    on global metrics or limited threshold checks to inform allocation
    decisions.
\end{enumerate}

Overall, the selection of an allocation method should consider the specific
constraints and demands of the environment. In distributed systems, combining
the $d$-choice and ($1+\beta$)-choice strategies may offer an optimal trade-off
between load balancing efficiency and adaptability. For systems requiring
minimal communication overhead, the Partial Information strategy serves as a
viable option, ensuring effective load distribution with limited data access.
Ultimately, a flexible approach that adapts to system constraints and
information availability can provide the best results for balanced allocations
in dynamic, high-demand settings.

\end{document}
